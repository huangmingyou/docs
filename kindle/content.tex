\clearpage
\begin{center}
\scalebox{3}{中华人民共和国演义}
\end{center}
\par
{\fzbys
\Large
话说我中华大地,本是文明古国。自远古巨鹿之战后,黄帝、炎帝入主中原,经略南方,风化渐开,遂奠定中华民族五千年文明的基础。我们都自称是炎黄子孙,道理就在这里。后来尧舜禅让,大禹治水,百姓安宁,生产发展,遂出现了夏商周三个王朝,开始了中华民族的文明史。春秋战国时,诸候问鼎,逐鹿中原。这时远处西陲的秦国引进人才,改革图强。范睢进策,商鞅变法。秦国人人勇于公战,怯于私斗,竟练成百万雄师。再加上君臣齐心,将士用命,遂灭六国,统一天下,废除分封,设立郡县。后世效之,遂成定制。再经一千八百年的发展,成为人口众多,物产丰富的诗书之乡,礼仪之邦。国居中央,四方来朝。古丝道上,驼铃叮?。茫茫大海,舟楫竞渡。火药、指南针、印刷术、造纸术四大发明传往世界各地,开辟了世界各民族进化之路。
}
