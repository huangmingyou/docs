\documentclass[UTF8]{ctexart}

% 包配置

\usepackage{algorithm}
\usepackage{algorithmic}
\usepackage{listings} %列出代码%
\usepackage{xcolor} %列出代码%
\usepackage{fontawesome}
\usepackage{graphicx}
\usepackage{geometry}
\usepackage{fontspec}
\usepackage{fancyhdr}
\usepackage[colorlinks,
            linkcolor=black,
            anchorcolor=blue,
            citecolor=green
            ]{hyperref}
%\usepackage{xeCJK}%中文字体


% 页面配置

%\geometry{papersize={40cm,15cm}}
\geometry{left=1cm,right=2cm,top=3cm,bottom=4cm}
\graphicspath{ {./img/} }


% 
\pagestyle{fancy}
\lhead{\author}
\chead{\date}
\lfoot{}
\cfoot{\thepage}
\rfoot{}

%\lstset{language=Ruby ,label=ruby,backgroundcolor=\color{gray!50},frame=shadow,breaklines=true,basicstyle=\small}

\lstset{
 columns=fixed,       
 numbers=left,                                        % 在左侧显示行号
 numberstyle=\tiny\color{gray},                       % 设定行号格式
 frame=none,                                          % 不显示背景边框
 backgroundcolor=\color[RGB]{245,245,244},            % 设定背景颜色
 keywordstyle=\color[RGB]{40,40,255},                 % 设定关键字颜色
 numberstyle=\footnotesize\color{darkgray},           
 commentstyle=\it\color[RGB]{0,96,96},                % 设置代码注释的格式
 stringstyle=\rmfamily\slshape\color[RGB]{128,0,0},   % 设置字符串格式
 showstringspaces=false,                              % 不显示字符串中的空格
 language=bash,                                        % 设置语言
}


\renewcommand{\headrulewidth}{0.4pt}
\renewcommand{\headwidth}{\textwidth}
\renewcommand{\footrulewidth}{0pt}
\renewcommand{\baselinestretch}{1.4}

% 英文字体配置部分
\setmainfont{Cabin}%Times New Roman
\setsansfont{Cabin Regular}
\setmonofont{Cabin Italic}
% 中文字体配置部分
\setCJKmainfont{SimSun}%正文字体
\setCJKsansfont{微软雅黑}%无衬线字体
\setCJKmonofont{SimSun}%等宽字体
\setCJKfamilyfont{boldsong}{Source Han Serif SC Heavy}

% 字体配置
\setCJKmainfont{SimSun}
\newCJKfontfamily\msyh{微软雅黑}


%\sffamily 无衬线字体
%\ttfamily 等宽字体
%\sffamily in English typing
%\ttfamily in English typing

% 标题格式
\ctexset {
section/name = {第,节}
}

\title{GNU/Linux入门}
\author{huangmingyou@gmail.com}
\date{\today}







% 正文
\begin{document}
\maketitle
\tableofcontents



\newpage
\section{\msyh 什么是GNU/Linux}
\subsection{\msyh 什么是GNU}
GNU是一个自由的操作系统,其内容软件完全以GPL方式发布。这个操作系统是GNU计划的主要目标,名称来自GNU's Not Unix!的递归缩写,因为GNU的设计类似Unix,但它不包含具著作权的Unix代码。GNU的创始人,理查德·马修·斯托曼,将GNU视为“达成社会目的技术方法”。

\begin{center}
\includegraphics[width=0.15\textwidth]{gnu.png}
\par
GNU logo
\end{center}
\par
作为操作系统,GNU的发展仍未完成,其中最大的问题是具有完备功能的内核尚未被开发成功。GNU的内核,称为Hurd,是自由软件基金会发展的重点,但是其发展尚未成熟。在实际使用上,多半使用Linux内核、FreeBSD等替代方案,作为系统核心,其中主要的操作系统是Linux的发行版。Linux操作系统包涵了Linux内核与其他自由软件项目中的GNU组件和软件,可以被称为GNU/Linux。


\subsection{\msyh 什么是Linux}
Linux内核(英语:Linux kernel)是一种开源的类Unix操作系统宏内核。整个Linux操作系统家族基于该内核部署在传统计算机平台(如个人计算机和服务器,以Linux发行版的形式)和各种嵌入式平台,如路由器、无线接入点、专用小交换机、机顶盒、FTA接收器、智能电视、数字视频录像机、网络附加存储(NAS)等。工作于平板电脑、智能手机及智能手表的Android操作系统同样通过Linux内核提供的服务完成自身功能。尽管于桌面电脑的占用率较低,基于Linux的操作系统统治了几乎从移动设备到主机的其他全部领域。截至2017年11月,世界前500台最强的超级计算机全部使用Linux。


\par
\begin{center}
\includegraphics[width=0.15\textwidth]{tux.png}
\par
Linux logo
\end{center}
\par
Linux内核最早是于1991年由芬兰黑客林纳斯·托瓦兹为自己的个人电脑开发的,他当时在Usenet新闻组comp.os.minix登载帖子,这份著名的帖子标志着Linux内核计划的正式开始。如今,该计划已经拓展到支持大量的计算机体系架构,远超其他操作系统和内核。它迅速吸引了一批开发者和用户,利用它作为其他自由软件项目的内核,如著名的 GNU 操作系统。而今天,Linux 内核已接受了超过1200家公司的近12000名程序员的贡献,其中包括一些知名的软硬件发行商。

从技术上说,Linux 只是一个符合POSIX 标准的内核。它提供了一套应用程序接口(API),通过接口用户程序能与内核及硬件交互。仅仅一个内核并不是一套完整的操作系统。有一套基于 Linux 内核的完整操作系统叫作Linux 操作系统,或是GNU/Linux(在该系统中包含了很多GNU 计划的系统组件)。

\subsection{\msyh 什么是GNU/Linux发行版}
Linux 发行版(英语:Linux distribution,也被叫做GNU/Linux 发行版),为一般用户预先集成好的Linux操作系统及各种应用软件。一般用户不需要重新编译,在直接安装之后,只需要小幅度更改设置就可以使用,通常以软件包管理系统来进行应用软件的管理。Linux发行版通常包含了包括桌面环境、办公包、媒体播放器、数据库等应用软件。这些操作系统通常由Linux内核、以及来自GNU计划的大量的函数库,和基于X Window的图形界面。有些发行版考虑到容量大小而没有预装 X Window,而使用更加轻量级的软件,如:busybox, UClibc 或 dietlibc。现在有超过300个Linux发行版(Linux发行版列表)。大部分都正处于活跃的开发中,不断地改进。
\par

由于大多数软件包是自由软件和开源软件,所以Linux发行版的形式多种多样——从功能齐全的桌面系统以及服务器系统到小型系统(通常在嵌入式设备,或者启动软盘)。除了一些定制软件(如安装和配置工具),发行版通常只是将特定的应用软件安装在一堆函数库和内核上,以满足特定用户的需求。

\par
这些发行版可以分为商业发行版,比如Ubuntu(Canonical公司)、Fedora(Red Hat)、openSUSE(Novell)和Mandriva Linux;和社区发行版,它们由自由软件社区提供支持,如Debian和Gentoo;也有发行版既不是商业发行版也不是社区发行版,如Slackware。
\newpage
\par
\par


\section{\msyh 如何获得GNU/Linux}

选择好发行版以后,从官方网站或者中国的镜像网站找到下载地址,下载最新的安装镜像,安装镜像通常是打包成iso格式。下载好安装镜像以后,用刻录软件把iso文件刻录到U盘或者DVD上。如果刻录到DVD还需要一个DVD刻录机,因此最方便的选择是刻录到U盘。如果只是体验Linux系统,
可以选择安装virtualbox这样的虚拟器来安装Linux。在Windows10系统上,还可以从windows软件商店安装ubuntu系统。这是一个不带图形界面的linux系统。
\par
除了自行下载安装镜像安装linux这种方法以外。还可以直接购买预装linux系统的电脑。也可以在云服务器厂商购买安装linux的云服务器。

\par

关于如何安装Linux系统,通常发行版官方的文档都会提供非常详细的安装说明。值得注意的是,如果希望安装Linux,Windows双系统。需要处理好磁盘分区。

\subsection{\msyh 安装linux系统涉及到的知识点}
\subsubsection{\msyh 硬盘分区}
通常,在安装windows系统的时候,需要对磁盘划分一个C盘,然后把windows系统安装到C盘,剩余的空间可以划分给其他例如D,E,F盘。安装完成,系统就能跑起来了。但是,安装Linux的时候,通常安装程序会让你划分分区,并且提示你把分区挂载到根目录或者/home目录。如果刚接触到Linux,会对挂载分区一头雾水。
\par


操作系统使用物理硬盘上的空间,一般会对硬盘进行分区(也可以不分区),操作系统会把分区信息映射到操作系统对应的标记符。windows的方法是一个分区映射到一个盘符,比如C,D,E,F,G。简单的一一对应关系。Linux的做法和windows有区别,Linux上没有盘符的概念,取而代之的是一棵目录树的概念。目录的根叫做根目录,用"/"表示。根目录下可以有一个叫做"home"的目录,表示成"/home",home目录下可以有一个叫做guest的目录,表示成/home/guest。根目录必须有一个对应的硬盘分区,术语叫做把某分区挂载到根目录。也可以把其他分区挂载到/home目录,或者/home/guest目录。
\par
当插入一块u盘到windows的时候,windows会分配一个新的盘符给u盘,比如G盘。Linux系统插入U盘的时候,现代的操作系统会挂载到类似/media/guest/ABCD123 这样的目录下。

\par
Linux系统有一个特性,叫做“一切皆文件”。或者说,任何东西在linux系统上,都能找到对应的文件。比如一个进程,一个tcp链接,一个硬盘,一个硬盘分区,一个鼠标等等。
\par
在windows系统下,硬盘分区和系统盘符通常是混合在一起的概念,说C盘的时候,既是在说操作系统里面的C盘,也同时暗指了硬盘上的一个分区。但是,在Linux系统下,首先会对硬盘命名,同时也会分区命名,并且会分配一个文件来与之对应。例如在linux上挂载一块scsi硬盘,会分配一个类似/dev/sda这样的名字来对应整个硬盘。而/dev/sda1对应这个硬盘的第一个分区,/dev/sda2 对应第2个分区。
我们可以把/dev/sda2这个分区挂载到/home目录,也同时可以把/dev/sda3挂载到/home/guest目录。

\subsubsection{\msyh 引导系统}
计算机开机通电以后,主板上的bios系统会完成自检,然后根据bios设置,选择从什么地方启动系统。可选择从光盘,U盘,硬盘,或者是网络启动系统。从硬盘启动的时候,会读取硬盘主分区表的启动信息。执行硬盘上的启动程序。硬盘分区表常见的有MBR和GPT,windows系统一般使用MBR。mac os 使用GPT。 Linux两种分区表都可以使用。Linux安装程序在安装过程中会询问把启动程序安装到什么设备上。假如一台计算机有一块scsi硬盘,选择安装到/dev/sda就行。注意,这里是/dev/sda,而不是/dev/sda1。表示启动程序是安装到硬盘的主分区上。早期的Linux使用lilo作为引导程序,现在流行的引导程序是grub2。

\subsubsection{\msyh 桌面系统}
习惯了windows只有一个桌面系统,首次接触到linux,会发现linux可以选择各种不同风格的桌面系统。先对几个相关术语进行解释。
\par
{\msyh X window 核心协议:}
\par
一个C/S模式的协议。服务器端包括键盘,鼠标,显示器。客户端包括各种应用程序。例如我在电脑A运行了一个X系统。并且在网络上监听。那么其他电脑就可以通过网络在电脑A上显示内容。X window 协议只是一个协议。具体的实现包括Xfree86,以及xorg等。现代的linux发行版通常都是用xorg。
\begin{center}
\includegraphics[width=0.15\textwidth]{x11.png}
\includegraphics[width=0.15\textwidth]{xorg.png}
\includegraphics[width=0.15\textwidth]{xfree86.png}
\par
X与             xorg与              xfree86 logo
\end{center}
\par
{\msyh Wayland协议:}
\par
同X window 协议竞争的新的协议。实现的实例叫做Weston。 ubuntu发行版可以选择使用wayland来取代X window协议。
\begin{center}
\includegraphics[width=0.15\textwidth]{wayland.png}
\par
wayland logo
\end{center}
\par
{\msyh 显示管理器:}
通常是指登录到桌面系统之前让你输入用户名和密码的那个组件。是的,在Linux系统里面,这样的组件也是独立的,而且有很多可以选择的项目。比如lightdm,xdm,gdm,wdm等。登录的时候,你可以选择登录gnome,kde,xface这样的桌面系统。
\par
{\msyh 桌面环境:}
为了方便用户使用计算机,一个典型的桌面环境会提供给用户文件管理,工具栏,热插拔设备管理,网络浏览器,邮件客户端等组件。当电脑启动登录进windows系统以后,看到的那个环境就是一个桌面环境。在Linux系统中,存在如gnome,kde这样功能比较全面的桌面环境,也存在xfce这样的轻量桌面环境。还存在专门给儿童使用的桌面环境。
\par
{\msyh 窗口管理器:}
管理窗口行为的系统,比如管理窗口如何最大,最小,怎么平铺等等。比如fvwm就是一个窗口管理器。可以通过配置窗口管理器,实现灵活的窗口操作。比如可以按F1让窗口最大或者按F2让窗口缩放到一个固定位置等等。
\end{document}
